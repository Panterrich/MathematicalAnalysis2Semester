\documentclass[a4paper,11.5pt]{article} % тип документа


%%%Библиотеки
	%\usepackage[warn]{mathtext}	
	%\usepackage[T2A]{fontenc} % кодировка
	\usepackage[utf8]{inputenc} % кодировка исходного текста
	\usepackage[english,russian]{babel} % локализация и переносы
	\usepackage{caption}
	\usepackage{listings}
	\usepackage{amsmath,amsfonts,amssymb,amsthm,mathtools}
	\usepackage{wasysym}
	\usepackage{graphicx}%Вставка картинок правильная
	\usepackage{float}%"Плавающие" картинки
	\usepackage{wrapfig}%Обтекание фигур (таблиц, картинок и прочего)
	\usepackage{fancyhdr} %загрузим пакет
	\usepackage{lscape}
	\usepackage{indentfirst}
	\usepackage{xcolor}
	\usepackage[normalem]{ulem}
	\usepackage{hyperref}

%%%Конец библиотек




%%%Настройка ссылок
	\hypersetup
	{
		colorlinks=true,
		linkcolor=blue,
		filecolor=magenta,
		urlcolor=blue
	}
%%%Конец настройки ссылок


%%%Настройка колонтитулы
	\pagestyle{fancy}
	\fancyhead{}
	\fancyhead[L]{Билет 12}
	\fancyhead[R]{Билеты Матан}
	\fancyfoot[C]{\thepage}
%%%конец настройки колонтитулы



							\begin{document}
						%%%%Начало документа%%%%


%%%Начало титульника
\begin{titlepage}

	\newpage
	\begin{center}
		\normalsize Московский физико-технический институт \\(госудраственный 			университет)
	\end{center}

	\vspace{6em}

	\begin{center}
		\Large Билеты к экзамену по матану [2 семестр]\\
	\end{center}

	\vspace{1em}

	\begin{center}
		\large \textbf{Билет 12}
	\end{center}

	\vspace{2em}

	\begin{center}
		\large Талашкевич Даниил Александрович\\
		Группа Б01-009
	\end{center}

	\vspace{\fill}

	\begin{center}
	Долгопрудный \\14.05.2021
	\end{center}
	
\end{titlepage}
%%%Конец Титульника



%%%Настройка оглавления и нумерации страниц
	\thispagestyle{empty}
	\newpage
	\tableofcontents
	\newpage
	\setcounter{page}{1}
%%%Настройка оглавления и нумерации страниц


					%%%%%%Начало работы с текстом%%%%%%

\newcommand{\eqdef}{\stackrel{\mathrm{def}}{=}}

\section*{Билет номер 12}

\subsection*{Степенные ряды с действительными членами}

\textbf{Теорема}. Если $R$ -- радиус сходимости степенного ряда и выполнено следующее:

\begin{equation*}
\Large \displaystyle \sum\limits_{k = 0}^{\infty} a_k(x-a)^k = f(x),\ x\in (a - R, a + R),\ a_k,a \in \mathbb(R)
\end{equation*}

то
\begin{enumerate}
\item $f$ бесконечно дифференцируемая функция на $(a-R,a+R)$ и  выполняется: 

\begin{equation*}
\Large \displaystyle f^{(m)}(x) = \sum\limits_{k = m}^{+\infty} k(k-1)\dots (k - (m-1))a_k(x-a)^{k-m}
\end{equation*}

\item $\Large \displaystyle \forall\ x \in (a - R, a + R) \mapsto \int\limits_{a}^{x} f(t)dt = \sum\limits_{k = 0}^{\infty} \frac{a_k}{k+1}(x-a)^{k+1}$ 

\end{enumerate} 

\textbf{Доказательство}. Will be ASAP.

\textbf{Следствие}. $a_n = \frac{f^{(k)}(a)}{k!}$


\subsection*{Бесконечная дифференцируемость суммы степенного ряда на интервале сходимости}

Покажем, что сумма степенного ряда дифференцируема в интервале сходимости.

\textbf{Теорема}. Сумма степенного ряда $f(x)=\sum_{n=0}^{\infty} c_{n}\left(x-x_{0}\right)^{n}$ дuфференцируема в интервале сходимости и производная равна
\begin{equation*}
f^{\prime}(x)=\sum_{n=1}^{\infty} n c_{n}\left(x-x_{0}\right)^{n-1},
\end{equation*}

причём ряды $\sum_{n=1}^{\infty} n c_{n}\left(x-x_{0}\right)^{n-1}$ и $\sum_{n=0}^{\infty} c_{n}\left(x-x_{0}\right)^{n}$ имеют одинаковый радиус сxoдимости.

\textbf{Доказательство}. Члены ряда $c_{n}\left(x-x_{0}\right)^{n}$ являются непрерывно дифференцируемыми на всей числовой прямой функциями. Пусть $R=\frac{1}{\varlimsup_{n \rightarrow \infty} \sqrt[n]{\left|c_{n}\right|}}$ радиус сходимости ряда $\sum_{n=0}^{\infty} c_{n}\left(x-x_{0}\right)^{n}$ и точка $x$ принадлежит интервалу сходимости $\left(x_{0}-R, x_{0}+R\right) .$ Тогда существует отрезок $[a, b] \subset\left(x_{0}-R, x_{0}+R\right)$, включающий точку $x .$

Рассмотрим ряд $\sum_{n=1}^{\infty} n c_{n}\left(x-x_{0}\right)^{n-1}$, полученный почленным дифференцированием ряда $\sum_{n=0}^{\infty} c_{n}\left(x-x_{0}\right)^{n} .$ Вычислим его радиус сходимости $R^{\prime}$

\begin{equation*}
R^{\prime}=\frac{1}{\varlimsup_{n \rightarrow \infty} \sqrt[n-1]{\left|n c_{n}\right|}} = \frac{1}{\varlimsup_{n \rightarrow \infty} \sqrt[n-1]{\left|c_{n}\right|}\cdot \sqrt[n-1]{n}} = \frac{1}{\varlimsup_{n \rightarrow \infty}(\left|c_{n}\right|^{\frac{1}{n}})^{\frac{n}{n-1}}}=R
\end{equation*}

Таким образом, ряды $\sum_{n=1}^{\infty} n c_{n}\left(x-x_{0}\right)^{n-1}$ и $\sum_{n=0}^{\infty} c_{n}\left(x-x_{0}\right)^{n}$ имеют одинаковый интервал сходимости, и, следовательно, на отрезке $[a, b]$ ряд $\sum_{n=1}^{\infty} n c_{n}\left(x-x_{0}\right)^{n-1}$ сходится равномерно. По теореме о дифференцируемости суммы функционального ряда сумма степенного ряда $f(x)$ дифференцируема в точке $x$ и верна формула

\begin{equation*}
f^{\prime}(x)=\sum_{n=1}^{\infty} n c_{n}\left(x-x_{0}\right)^{n-1}
\end{equation*}

что полностью доказывает теорему. $\square$

Теперь в силу доказанной теоремы при дифференцировании суммы степенного ряда вновь получаем степенной ряд с тем же радиусом сходимости. Это позволяет нам сформулировать
следующую теорему:

\textbf{Теорема}. Сумма степенного ряда $f(x)=\sum\limits_{n=0}^{\infty} c_{n}\left(x-x_{0}\right)^{n}$ дuффepeнцируема любое количество раз и верна формула

\begin{equation*}
f^{(k)}(x)=\sum_{n=k}^{\infty} c_{n} n(n-1)(n-2) \ldots(n-k+1)\left(x-x_{0}\right)^{n-k}
\end{equation*}

причём радиусы сходимости всех получающихся рядов одинаковы.

\textbf{Доказательство}. По предыдущей теореме функция $f(x)=\sum\limits_{n=0}^{\infty} c_{n}\left(x-x_{0}\right)^{n}$ дифференцируема и $f^{\prime}(x)=\sum\limits_{n=1}^{\infty} n c_{n}\left(x-x_{0}\right)^{n-1}$, причём радиусы сходимости обоих рядов совпадают. Далее, пусть существует

\begin{equation*}
f^{(k-1)}(x)=\sum_{n=k-1}^{\infty} c_{n} n(n-1)(n-2) \ldots(n-k+2)\left(x-x_{0}\right)^{n-k+1}
\end{equation*}

Применяя к функции $f^{(k-1)}(x)$ предыдущую теорему, получаем, что $f^{(k-1)}(x)$ дифференцируема и верна формула

\begin{equation*}
f^{(k)}(x)=\left(f^{(k-1)}(x)\right)^{\prime}=\sum_{n=k}^{\infty} c_{n} n(n-1)(n-2) \ldots(n-k+2)(n-k+1)\left(x-x_{0}\right)^{n-k},
\end{equation*}

причём радиусы сходимости рядов для $f^{(k-1)}(x)$ и $f^{(k)}(x)$ совпадают. Тем самым, следуя методу математической индукции, полностью доказывает эту теорему. $\square$


\subsection*{Единственность представления функции степенным рядом}

\textbf{Определение.} Регулярная функция.

Пусть в каждой точке $z \in \mathbb{E}$, где $\mathbb{E}$ -- множество точек комплексной плоскости, поставлено в соответствие комплексное число $\omega$. На множестве $\mathbb{E}$ определена функция комплексного переменного, $\omega = f(z)$.

Если $\forall \epsilon > 0\ \exists \ \sigma = \sigma_{\epsilon} > 0:\ \forall z\ :\ |z - a| < \sigma_{\epsilon} \longmapsto |f(z) - f(a)| < \epsilon$, то функцию $f(z)$ называют непрерывной в точке а.

И , наконец, Функция комплексного переменного $f(z)$ называется регулярной в точке $a$, если она определена в некоторой окрестности точки $a$ и представима в некотором круге $|z - a| < \rho$, $\rho > 0$, сходящимся к $f(z)$ степенным рядом $f(z) = \sum\limits_{n = 0}^{\infty} c_n(z-a)^n\ \ $ (*).

\textbf{Теорема}. Единственность представления функции степенным рядом.

Функция $f(z)$, регулярная в точке $a$, единственным образом представляется рядом (*).

\textbf{Доказательство}. Пусть функция $f(z)$ имеет два представления в виде степенного ряда в круге $K = \{z: |z-a|<\rho\}$, где $\rho > 0$, т.е.

\begin{equation*}
f(z) = \sum\limits_{n = 0}^{\infty}c_n(z-a)^n = \sum\limits_{n = 0}^{\infty}\widetilde{c_n}(z-a)^n\ \ (**)
\end{equation*}

Теперь покажем, что $c_n = \widetilde{c_n} \ $, для $n = 0, 1, 2,\dots$

По условию ряды $\sum\limits_{n = 0}^{\infty}c_n(z-a)^n$ и $\sum\limits_{n = 0}^{\infty}\widetilde{c_n}(z-a)^n$ сходятся в круге $K$, и поэтому эти ряды сходятся равномерно в круге $K_1 = \{z: |z - a|\leqslant \rho_1 < \rho \}$, а их общая сумма -- непрерывная в круге $K_1$ функция.В частности, функция $f(z)$ непрерывна в точке $a$. Подходя к пределу при $z \to a$ в равенстве (**), получаем $c_0 = \widetilde{c_0}$. Отбрасывая одинаковые слагаемые $c_0$ и $\widetilde{c_0}$ в равенстве (**), получаем после деления на $(z - a)$ равенство:

\begin{equation*}
c_1 + c_2(z - a) + c_3(z - a)^2 +\ \dots = \widetilde{c_1} + \widetilde{c_2}(z - a) + \widetilde{c_3}(z - a)^2 +\ \dots\ ,
\end{equation*}

которое справедливо в круге $K$ с выколотой точкой $a$. Ряды в левой и правой части сходятся равномерно в круге $K_1$. Переходя в равенстве к пределу при $z \to a$, получаем $c_1 = \widetilde{c_1}$. Справедливость равенства $c_n = \widetilde{c_n}$ при любой $n \in \mathbb(N)$ устанавливается при помощи индукции.


\subsection*{Достаточные условия разложимости бесконечно дифференцируемой функции в степенной ряд}

\textbf{Теорема}. Достаточные условия сходимости ряда Тейлора к функции.

Если $f$ бесконечно дифференцируемая функция на ($a - \delta , a + \delta$), $\delta > 0$ и $\exists M > 0 : \forall x \in (a - \delta, a + \delta) \mapsto |f^{(k)}(x)| \leqslant M\ ,\ k = 0,1,\dots \ $, то ряд Тейлора сходится к функции $f(x)$ в каждой точке $x$ нашего интервала:

\begin{equation*}
 \Large \displaystyle f(x) =f(a) + \sum\limits_{k = 1}^{\infty} \frac{f^{(k)}(a)}{k!}(x-a)^k\ ,\ \forall x\in (a - \delta, a + \delta)
\end{equation*}

\textbf{Доказательство}. Достаточные условия разложимости бесконечно дифференцируемой функции в степенной ряд.

\begin{equation*}
\begin{gathered}
	\mathbf{r}_n(x) = \frac{f^{(n+1)}(\xi)}{(n+1)!}(x-a)^{n+1}\ ,\ \text{где } \xi \text{ между } x \text{ и } a  \\
	|\mathbf{r}_n(x)| \leqslant M\cdot \frac{|x - a|^{n+1}}{(n+1)!}\\
	\text{т.к.  }|x - a|\ \geqslant 0 \Rightarrow \lim_{k \to \infty} \frac{|x - a|^k}{k!} = 0\ ,\ \text{тогда справедливо следующее:}\\
	\forall x \in (a - \delta, a + \delta)\ \ \forall n\in \mathbb{N} \longmapsto |\mathbf{r}_n(x)| 	\leqslant M \cdot \frac{\mid x - a\mid^{n+1}}{(n + 1)!} \underset{n \rightarrow \infty} \longrightarrow 0\ \ \ \square\ .
\end{gathered}
\end{equation*}


\subsection*{Ряд Тейлора}

Пусть функция $f$ -- бесконечно дифференцируема в точке $a$ (т.е в этой точке у функции $f$ существует производная любого порядка), тогда

\textbf{Определение}. Рядом Тейлора функции $f$ в точке $a$ называется следующее выражение:
\begin{equation*}
f(a) + \sum\limits_{k = 1}^{\infty} \frac{f^{(k)}(a)}{k!}(x-a)^k 
\end{equation*}

\textbf{Замечание}. Если функция регулярна в точке $a$, то она раскладывается в степенной ряд и этот степенной ряд и есть ряд Тейлора, однако не все функции раскладываются в степенной ряд, поэтому справедливо следующее выражение:
\begin{equation*}
f(x) \neq f(a) + \sum\limits_{k = 1}^{\infty} \frac{f^{(k)}(a)}{k!}(x-a)^k 
\end{equation*}

\textbf{Пример}. Рассмотрим следующую функцию:

\begin{equation*}
f(x) = \begin{cases}
   e^{-\frac{1}{x^2}}\ ,\ x \neq 0;\\
   0\ ,\ x = 0\ .
 \end{cases}
\end{equation*}

Эта функция непрерывная в нуле. Найдем ее производные:

\begin{equation*}
f^{\text{'}}(x) = \frac{2}{x^3}\cdot e^{-\frac{1}{x^2}}\\
\end{equation*}

\begin{equation*}
f^{\text{''}}(x) = \left[ \left( \frac{2}{x^3}\right)^2 - \frac{6}{x^4} \right]\cdot e^{-\frac{1}{x^2}}\\
\end{equation*}

\begin{equation*}
f^{\text{'''}}(x) = \left[ \left( \frac{2}{x^3}\right)^3 - \frac{12}{x^7} - \frac{2^4}{x^4} + \frac{24}{x^5} \right]\cdot e^{-\frac{1}{x^2}}\\
\end{equation*}

Таким образом $f^{\text{(m)}}(x) = Q_{3m}(\frac{1}{x})\cdot e^{-\frac{1}{x^2}}$, где $Q_{3m}(\frac{1}{x})$ -- многочлен степени $3m$ от $\frac{1}{x}$. Тогда понятно, что 

\begin{equation*}
\lim_{x\to 0} \frac{e^{-\frac{1}{x^2}}}{x^k} = 0 \Rightarrow 
\end{equation*}

\begin{equation*}
\Rightarrow f^{\text{(m)}}(x) = \begin{cases}
   Q_{3m}(\frac{1}{x}) \cdot e^{-\frac{1}{x^2}}\ ,\ x \neq 0;\\
   0\ ,\ x = 0\ .
 	\end{cases}
\end{equation*}

Тогда $\forall x \neq a$ ряд Тейлора будет представлять собой нулевой ряд, хотя сама функция не нулевая $\Rightarrow f(x) \neq f(a) + \sum\limits_{k = 1}^{\infty} \frac{f^{(k)}(a)}{k!}(x-a)^k.\ \square$ 


\subsection*{Формула Тейлора с остаточным членом в интегральной форме.}

Функция $f$ -- бесконечно дифференцируема в некоторой окрестности точки $a$, тогда этой функции соответствует некоторый ряд: 

\begin{equation*}
\Large \displaystyle f(a) + \sum\limits_{k = 1}^{\infty} \frac{f^{(k)}(a)}{k!}(x-a)^k
\end{equation*}


\textbf{Обозначение}. $\Large \displaystyle P_n(x) = f(a) + \sum\limits_{k = 1}^{n} \frac{f^{(k)}(a)}{k!}(x-a)^k$ -- $n$-ая частичная суммма ряда Тейлора (многочлен Тейлора).

Тогда, если $\mathbf{r}_n(x) = f(x) - P_n(x) \underset{n \rightarrow \infty} \longrightarrow 0$, то это означает, что ряд Тейлора сходится к функции $f$ в точке $x$:

\begin{equation*}
\Large \displaystyle f(x) = f(a) + \sum\limits_{k = 1}^{\infty} \frac{f^{(k)}(a)}{k!}(x-a)^k
\end{equation*}

\textbf{Теорема}. Если $f^{(n+1)}$ непрерывна на $(a - \delta, a + \delta),\ \delta > 0$, то:

\begin{enumerate}
\item $\Large \displaystyle \mathbf{r}_n(x) = \frac{1}{n!} \int\limits_{a}^{x} (x - t)^nf^{(n+1)}(t)dt$, т.е. её остаточный член на этом интервале представим в интегральной форме.
\item $\Large \displaystyle \mathbf{r}_n(x) = \frac{f^{(n+1)}(\xi)}{(n+1)!}(x-a)^{n+1}$
\end{enumerate}

\textbf{Доказательство}. 
\begin{enumerate}
	\item Доказательство будем проводить при помощи мат. индукции:
	\begin{enumerate}
		\item Мы знаем, что $\Large \displaystyle f(x) - f(a) = \int\limits_{a}^{x} f^{'}(t)dt$. Тогда:
		\begin{equation*}
		\begin{cases}
  		u = f^{'}(t)\ ,\ dv = dt\\
  		du = f^{''}(t)dt\ ,\ v = -(x-t), \text{ x - это константа}
 		\end{cases}
		\end{equation*}
		получаем, что $\Large \displaystyle f(x) - f(a) = -f^{'}(t)(x-t) \bigg|_a^x + \int\limits_a^x (x-t) f^{''}(t)dt = $\\
		$\Large \displaystyle = f^{'}(a)(x-a) + \frac{1}{1!} \int\limits_a^x (x-t) f^{''}(t)dt \Rightarrow$\\
		 $\Rightarrow \Large \displaystyle  f(x) = f(a) + \frac{f^{'}(a)}{1!}(x-a) + \frac{1}{1!} \int\limits_a^x (x-t) f^{''}(t)dt$.\\
	Получили при $n = 1$ остаточный член в интегральной форме (получена база индукции).
		\item Предположим, что при $n - 1$ верно, тогда найдем для $n$ :
		\begin{equation*}
			\Large \displaystyle f(x) = f(a) + \sum\limits_{k = 1}^{n - 1} \frac{f^{(k)}(a)}{k!}(x-a)^k + \frac{1}{(n-1)!}\int\limits_a^x (x-t)^{n - 1} f^{(n)}(t)dt
		\end{equation*}
		Тогда:
		\begin{equation*}
		\begin{cases}
  		u = f^{n}(t)\ ,\ dv = (x-t)^{n-1}dt\\
  		du = f^{n + 1}(t)dt\ ,\ v = -\frac{(x-t)^n}{n}
 		\end{cases}
		\end{equation*}
		получаем, что $\Large \displaystyle f(x) = f(a)\ +\ \sum\limits_{k = 1}^{n - 1} \frac{f^{(k)}(a)}{k!}(x-a)^k\ -\ \frac{(x-t)^n f^{(n)}(t)}{n!} \bigg|_a^x + +\ \frac{1}{n!} \int\limits_a^x (x-t)^n f^{(n+1)}(t)dt$. Тогда получаем, что:
	\begin{equation*}
		\Large \displaystyle f(x) = f(a)\ +\ \sum\limits_{k = 1}^{n} \frac{f^{(k)}(a)}{k!}(x-a)^k + \frac{1}{n!} \int\limits_a^x (x-t)^n f^{(n+1)}(t)dt \ \square
	\end{equation*}
		 
	\end{enumerate}

	\item Это просто остаточный член в форме Лагранжа (доказывалось в прошлом семестре).
\end{enumerate}




\end{document}
