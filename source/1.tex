\documentclass[a4paper,12pt]{article} % тип документа

%  Русский язык
\usepackage[T2A]{fontenc}			% кодировка
\usepackage[utf8]{inputenc}			% кодировка исходного текста
\usepackage[english,russian]{babel}	% локализация и переносы

\usepackage{graphicx}               % импорт изображений
\usepackage{wrapfig}                % обтекаемые изображения
\graphicspath{{pictures/}}          % обращение к подкаталогу с изображениями
\usepackage[14pt]{extsizes}         % для того чтобы задать нестандартный 14-ый размер шрифта
\usepackage{amsfonts}               % буквы с двойными штрихами
\usepackage[warn]{mathtext}         % русский язык в формулах
\usepackage{indentfirst}            % indent first
\usepackage[margin = 25mm]{geometry}% отступы полей
\usepackage{amsmath}                % можно выводить фигурные скобочки -- делать системы уравнений
\usepackage[table,xcdraw]{xcolor}   % таблицы
\usepackage{amsmath,amsfonts,amssymb,amsthm,mathtools} % Математика
\usepackage{wasysym}                % ???
\usepackage{upgreek}                % ???  

\usepackage{gensymb} % degree symbol
\usepackage{mathrsfs}

%Заговолок
\author{Филиппенко Павел}
\title{Билет 1.}
\date{\today}


\begin{document} % начало документа

\maketitle
\newpage

\section*{Билет 1.}

\noindent \textit{Введем понятия:}\\[0.5 cm]

\begin{enumerate}
    \item $\mathbb{R}^m$ -- m-мерное координатное пространство.
    \item $x = (x_1, x_2, \dots, x_m)$ -- точка m-мерного пространства, ~ \\ $x_j \in \mathbb{R}, ~
    j = 1, 2, \dots, m$ -- координата точки.
    \item $x, y \in \mathbb{R}^m$; $\rho_e(x, y) = \sqrt{\sum_{j = 1}^m (x_j - y_j)^2}$ -- расстояние между точками $x$ и $y$.
    \item $\mathbb{E}^m = (\mathbb{R}^m, \rho_e)$ -- m-мерное евклидово пространство.
    \item $\mathscr{M} = (M, \rho)$ -- метрическое пространство, где $M \subset \mathbb{R}^m$ -- некоторое множество, $\rho(x, y)$ -- функция, задающая расстояние между точками $x, y$ множества $M$ (метрика).
    \item $B_{\varepsilon}(x_0) = \{x \in \mathbb{E}^m : \rho(x, x_0) < \varepsilon \}$ -- m-мерный шар с центром в точке $x_0$ и радиусом $\varepsilon$ (шаровая $\varepsilon$-окресность точки $x_0$).
    \item $\text{П}_{r_1, r_2, \dots r_m}(x_0) = \{ x \in \mathbb{E}^m : |x_j - x_{0j}| < r_j, ~ j = 1, 2, \dots, m \}$ -- m-мерный прямоугольник с центром $x_0$ и сторонами $2r_1, 2r_2, \dots, 2r_m$.\\
    
    $\text{П}_r(x_0) = \{ x \in \mathbb{E}^m : |x_j - x_{0j}| < r \}$ -- m-мерный квадрат с центром в точке $x_0$ и стороной $2r$.
\end{enumerate}

\noindent \textbf{Предложение:}

\begin{enumerate}
    \item в любую шаровую окрестность можно вписать прямоугольную окрестность.
    \item в любую прямоугольную окрестность можно вписать шаровую окрестность.
\end{enumerate}

\noindent \textbf{Доказательство:} Пусть $B_{\varepsilon}(x_0), ~ \text{П}_{r}(x_0)$ -- шаровая и прямоуголная окрестности точки $x_0$, $r = (r_1, r_2, \dots, r_m)$. Везьмем $\delta = min\{r_1, r_2, \dots, r_m \}$, тогда:

\begin{enumerate}
    \item $\text{П}_{\delta}(x_0) \subset B_{\varepsilon}(x_0), ~~ \delta = \frac{\varepsilon}{\sqrt{m}}$
    \item $B_{\varepsilon}(x_0) \subset \text{П}_{\delta}(x_0), ~~ \delta = \varepsilon$
\end{enumerate}

\subsection*{Предел последовательности точек в n–мерном евклидовом пространстве.}

\noindent $\mathscr{M} = (M, \rho)$, $\{x^n \}_{n = 1}^{\infty}$ -- последовательность точек в $\mathscr{M}$.\\[0.5 cm]

\noindent \textbf{Определение:} Последовательность $\{x^n \} \subset \mathscr{M}$ точек метрического пространства сходится к точке $a \in \mathscr{M}$, если:
\begin{equation*}
    \lim_{n \to \infty} \rho(x^n, a) = 0
\end{equation*}

\[ [\lim_{n \to \infty} x^n = a] \stackrel{def}{=} [\forall \varepsilon > 0 ~ \exists N = N(\varepsilon) : \forall n \geqslant N \mapsto \rho(x^n, a) < \varepsilon] \]

\noindent Любой шар с центром в точке $a$ и радиусом $\varepsilon$ содержит все члены последовательности $\{x^n \}$ за исключением быть может конечного числа $N$.\\

\noindent \textbf{Лемма:} Сходящаяся последовательность точек ограничена.\\

\noindent \textbf{Доказательство:}

\[ \lim_{n \to \infty} x^n = a \stackrel{def}{\Rightarrow} \lim_{n \to \infty} y_n = 0 \
\]
\noindent где $y_n = \rho(x^n, a)$. Тогда $\{y_n \}$ -- бесконечно малая последовательность $\Rightarrow$ $\{y_n \}$ -- ограничена, т. е. $\exists ~ C > 0 ~ : ~ 0 \leqslant y_n = \rho(x^n, a) \leqslant C$.\\

\noindent \textbf{Лемма:} Сходящаяся последовательность точек имеет единственный предел.\\

\noindent \textbf{Доказательство:} Будем доказывать от протвного: предположим, что

\begin{equation*}
    \exists ~ a \neq b : 
    \begin{cases}
        \lim_{n \to \infty} x^n = a\\
        \lim_{n \to \infty} x^n = b
    \end{cases}
    \Rightarrow
\end{equation*}

\[ (1) ~~ \forall \varepsilon > 0 ~ \exists ~ N_1 = N_1(\varepsilon) : ~ \forall n \geqslant N_1 \mapsto \rho(x^n, a) < \frac{\varepsilon}{2} \]

\[ (2) ~~ \forall \varepsilon > 0 ~ \exists ~ N_2 = N_2(\varepsilon) : ~ \forall n \geqslant N_2 \mapsto \rho(x^n, b) < \frac{\varepsilon}{2} \]

\noindent Рассмотрим $N = \{N_1, N_2\}$, тогда:

\[ \forall \varepsilon > 0 ~ \exists ~ N ~ : ~ \forall n \geqslant N \mapsto \rho(a, b) \leqslant \rho(x^n, a) + \rho(x^n, b) < \varepsilon \]

\noindent Таким образом, получаем, что $\rho(a, b) = 0 ~ \Rightarrow ~ a \equiv b$ -- противоречие.

\subsection*{Теорема Больцано–Вейерштрасса и критерий Коши сходимости последовательности.}

\noindent \textbf{Определение:} Последовательность $\{x^n \} \subset \mathscr{M}$ ограничена, если $\exists R > 0 ~ \forall n \mapsto \rho(x^n, 0) \leqslant R$\\

\noindent \textbf{Теорема:} Пусть $\{x^n \} = \{x_1^n, x_2^n, x_3^n, \dots, x_m^n\} \subset \mathbb{E}^m$ -- последовательность точек m-мерного евклидового пространства, а $a = (a_1, a_2, \dots, a_m) \in \mathbb{E}^m$ -- точка m-мерного евклидового пространства, тогда

\[ x^n \xrightarrow[n \to \infty]{} a ~ \Leftrightarrow ~ \forall j ~ x_j^n \xrightarrow[n \to \infty]{} a_j \]

\noindent \textbf{Доказательство:} \textit{Необходимость} ($\Rightarrow$)\\

\noindent По условию дано:

\[ \forall \varepsilon > 0 ~ \exists N = N(\varepsilon) : ~ \forall n \geqslant N \mapsto \rho(x^n, a) < \varepsilon \]

\noindent Тогда:

\[ \rho(x^n, a) = \sqrt{\sum_{j = 1}^m (x_j^n - a_j)^2} \Rightarrow |x_j^n - a_j| \leqslant \rho(x^n, a) < \varepsilon \]

\noindent Таким образом получаем:

\[ [\forall \varepsilon > 0 ~ \exists N = N(\varepsilon) : ~ \forall n \geqslant N ~~ \& ~~ \forall j = 1, 2, \dots, m \mapsto |x_j^n - a_j| < \varepsilon] \stackrel{def}{=} \lim_{n \to \infty} x_j^n = a_j \]

\noindent \textit{Достаточность} ($\Leftarrow$)\\

\noindent Запишем определение покоординатной сходимости:

\[ \forall \varepsilon > 0 ~ \exists ~ N_j = N_j(\varepsilon) : ~ \forall n \geqslant N_j \mapsto |x_j - a_j| < \frac{\varepsilon}{\sqrt{m}} \]

\noindent Пусть $N = max\{N_1, N_2, \dots, N_m \} \Rightarrow \forall n \geqslant N \mapsto |x_j - a_j| < \frac{\varepsilon}{\sqrt{m}}$ для всех $j$.

\[ \rho(x^n, a) = \sqrt{\sum_{j = 1}^m (x_j^n - a_j)^2} < \sqrt{m \cdot \frac{\varepsilon^2}{m}} = \varepsilon \Rightarrow \]

\[ \lim_{n \to \infty} x^n = a \]

\noindent \textbf{Теорема [Теорема Больцано-Вейерштраса]:} Из любой ограниченной последовательности $\{x^n \} \subset \mathbb{E}^m$ можно выделить сходящуюся подпоследовательность $\{x^{k_n} \} \subset \mathbb{E}^m$.\\

\noindent \textbf{Доказательство:}\\
$\{x^n \} \subset \mathbb{E}^m$ является ограниченной $\stackrel{def}{\Rightarrow}$ $\exists R > 0 ~ \forall n \mapsto \rho_e(x^n, 0) \leqslant R$. Тогда для всех $j$ последовательность $\{x_j^n \}$ так же ограничена ($x_j^n$ -- $j$-я компонента).\\

\noindent $\{x_1^n \}$ ограничена, тогда по теореме Больцано-Вейерштраса для числовой последовательности существует подпоследовательность $\{x_1^{k_{n_1}} \}$ такая, что $x_1^{k_{n_1}} \xrightarrow[n_1 \to \infty]{} a_1$.\\

\noindent Возмем подпоследовательность $\{x^{k_{n_1}} \} \subset \mathbb{E}^m$ (по номерам $k_{n_1}$ выбираем из последовательности $\{x^n \}$ точки). И рассмотрим числовую последовательность $\{x_2^{k_{n_1}} \}$. Она ограничена (как подпоследовательность ограниченной последовательности) и следовательно, по теореме Больцано-Вейерштраса существует подпоследовательность $\{x_2^{k_{n_2}} \}$ такая, что $x_2^{k_{n_2}} \xrightarrow[n_2 \to \infty]{} a_2$. При этом все еще справедливо $x_1^{k_{n_2}} \xrightarrow[n_2 \to \infty]{} a_1$.\\

\begin{equation*}
    \{x^n \} \subset \mathbb{E}^m \Rightarrow \{x_1^n \} \Rightarrow \{x_1^{k_{n_1}} \} \Rightarrow x_1^{k_{n_1}} \xrightarrow[n_1 \to \infty]{} a_1
\end{equation*}

\begin{equation*}
    \{x^{k_{n_1}} \} \subset \mathbb{E}^m \Rightarrow \{x_2^{k_{n_1}} \} \Rightarrow \{x_2^{k_{n_2}} \} \Rightarrow 
    \begin{cases}
        x_2^{k_{n_2}} \xrightarrow[n_2 \to \infty]{} a_2\\
        x_1^{k_{n_2}} \xrightarrow[n_2 \to \infty]{} a_1
    \end{cases}    
\end{equation*}

\begin{equation*}
    \{x^{k_{n_2}} \} \subset \mathbb{E}^m \Rightarrow \{x_3^{k_{n_2}} \} \Rightarrow \{x_3^{k_{n_3}} \} \Rightarrow 
    \begin{cases}
        x_3^{k_{n_3}} \xrightarrow[n_3 \to \infty]{} a_3\\
        x_2^{k_{n_3}} \xrightarrow[n_3 \to \infty]{} a_2\\
        x_1^{k_{n_3}} \xrightarrow[n_3 \to \infty]{} a_1
    \end{cases}    
\end{equation*}

\[ \dots \]

\begin{equation*}
    \{x^{k_{n_{m-1}}} \} \subset \mathbb{E}^m \Rightarrow \{x_m^{k_{n_{m-1}}} \} \Rightarrow \{x_m^{k_{n_m}} \} \Rightarrow 
    \begin{cases}
        x_m^{k_{n_m}} \xrightarrow[n_m \to \infty]{} a_m\\
        \dots \\
        x_1^{k_{n_m}} \xrightarrow[n_m \to \infty]{} a_1
    \end{cases}    
\end{equation*}

\noindent Таким образом мы нашли подпоследовательность $\{x^{k_{n_m}} \} \subset \mathbb{E}^m$ такую, что $x^{k_{n_m}} \xrightarrow[n_m \to \infty]{} a = (a_1, a_2, \dots, a_m)$.\\

\noindent \textbf{Определение:} Последовательность $\{x^n \} \subset \mathbb{E}^m$ называется \textit{фундаменталной}, если $\forall \varepsilon > 0 ~ \exists N = N(\varepsilon) : ~ \forall n \geqslant N ~ \& ~ \forall k \geqslant N \mapsto \rho(x^n, x^k) < \varepsilon$\\

\noindent \textbf{Теорема [аналог критерия Коши]:} если $\{x^n \} \subset \mathbb{E}^m$ сходится, то она фундаментальна.\\

\noindent \textbf{Доказательство:}

\[ x^n \xrightarrow[n \to \infty]{} a \Rightarrow \forall \varepsilon > 0 ~ \exists N = N(\varepsilon) : \]

\[ \forall n \geqslant N \mapsto \rho(x^n, a) < \frac{\varepsilon}{2} \]
\[ \forall k \geqslant N \mapsto \rho(x^k, a) < \frac{\varepsilon}{2} \]

\[ \rho(x^n, x^k) \leqslant \rho(x^n, a) + \rho(x^k, a) < \varepsilon \]

\noindent \textbf{Замечание:} В отличие от классического критерия Коши обратное утверждение в теореме НЕВЕРНО!\\

\noindent \textbf{Контрпример:} Рассмотрим метрическое пространство $\mathscr{M} = (\mathbb{Q}, \rho)$, $\rho(x, y) = |x - y|$. Рассмотрим последовательность $x_1 = 1, ~ x_{n + 1} = \frac{1}{2}(x_n + \frac{2}{x_n})$. Данная последовательность является фундаментальной, но при этом не является сходящейся в $\mathscr{M}$.

\subsection*{Внутренние, предельные, изолированные точки множества.}

\noindent \textbf{Определение:} точка $x_0$ множества $X \subset \mathscr{M} = (M, \rho)$ называется \textit{внутренней точкой множества}, если существует $B_{r}(x_0)$ : $B_{r}(x_0) \subset X$.\\

\noindent \textbf{Определение:} точка $x_0$ множества $X \subset \mathscr{M} = (M, \rho)$ называется \textit{предельной точкой множества}, если любая окрестность точки $x_0$ содержит по крайней мере одну точку множества $X$, отличную от $x_0$.
\[ \forall \varepsilon > 0 ~ \exists ~ x_{\varepsilon} \in X : x_{\varepsilon} \neq x_0 ~~ \& ~~ x_{\varepsilon} \in B_{\varepsilon}(x_0) \]

\noindent \textbf{Определение:} точка $x_0$ множества $X \subset \mathscr{M} = (M, \rho)$ называется \textit{изолированной точкой множества}, если у этой точки существует окрестность, не содержащая никаких других точек множества $X$.
\[ \exists ~ r > 0 : \forall x \in B_r(x_0) : ~ x \neq x_0 \mapsto x \notin X \]

\noindent \textbf{Определение:} точка $x_0$ множества $X \subset \mathscr{M} = (M, \rho)$ называется \textit{точкой прикосновения множества}, если любая окрестность этой точки содержит по крайней мере одну точку множества $X$.

\[ \forall \varepsilon > 0 ~ \exists ~ x_{\varepsilon} \in X ~ : ~ x_{\varepsilon} \in B_{\varepsilon}(x_0) \]

\noindent \textbf{Определение:} точка $x_0$ множества $X \subset \mathscr{M} = (M, \rho)$ называется \textit{граничной точкой множества}, если в любой ее окрестности существуют точки, как принадлежащие множеству $X$, так и не принадлежащие ему.

\[ \forall \varepsilon > 0 ~ \exists ~ x_{\varepsilon}' \in X ~ \& ~ x_{\varepsilon}'' \notin X : x_{\varepsilon}', x_{\varepsilon}'' \in B_{\varepsilon}(x_0) \]

\subsection*{Открытые и замкнутые множества, их свойства.}

\noindent \textbf{Определение:} Множество $X \subset \mathscr{M}$ называется \textit{открытым}, если любая его точка внутренняя.\\

\noindent \textbf{Определение:} Множество $X \subset \mathscr{M}$ называется \textit{замкнутым}, если оно содержит все свои предельные точки.\\

\noindent \textbf{Теорема:} открытые множества метрического пространства $\mathscr{M}$ обладают следующими свойствами:

\begin{enumerate}
    \item $\mathscr{M}, \varnothing$ -- открытые множества.
    \item $\bigcup \limits_{\alpha \in A} X_{\alpha}$ -- объединение любого числа открытых множеств $X_{\alpha}$ есть открытое множество.
    \item $\bigcap \limits_{j = 1}^K X_j$ -- пересечение конечного числа открытых множеств $X_j$ есть открытое множество.
\end{enumerate}

\noindent \textbf{Доказательство свойства 2:} Возмем произвольную точку $x \in X = \bigcup \limits_{\alpha \in A} X_{\alpha}$, тогда существует множество $X_{\alpha_0}$, такое что $x \in X_{\alpha_0}$. Но $X_{\alpha_0}$ -- открытое множество (по условию), поэтому $\exists ~ \varepsilon_0 : B_{\varepsilon_0}(x) \subset X_{\alpha_0} \subset X$, таким образом, получается, что любая точка множества $X$ входит в него с некоторой $\varepsilon$-окрестностью, это значит, что $X$ -- открытое множество.\\

\noindent \textbf{Доказательство свойства 3:} Возмем произвольную точку $x \in X = \bigcap \limits_{j = 1}^K X_{j}$, тогда $x \in X_j, ~ j = 1, 2, \dots, K$. Но каждое $X_j$ -- открытое множество, поэтому $\forall ~ j ~ \exists ~ \varepsilon_j : B_{\varepsilon_j}(x) \subset X_j$. Пусть $\varepsilon = min \{\varepsilon_1, \varepsilon_2, \dots, \varepsilon_K \}$, тогда $\forall ~ j ~ B_{\varepsilon}(x) \subset X_j ~ \Rightarrow ~ B_{\varepsilon}(x) \subset X$, это значит, что $X$ -- открытое множество.\\

\noindent \textbf{Теорема:}

\begin{enumerate}
    \item $Y \backslash (\bigcap \limits_{\alpha \in A} X_{\alpha}) = \bigcup \limits_{\alpha \in A} (Y \backslash X_{\alpha})$
    \item $Y \backslash (\bigcup \limits_{\alpha \in A} X_{\alpha}) = \bigcap \limits_{\alpha \in A} (Y \backslash X_{\alpha})$
\end{enumerate}

\noindent \textbf{Доказательство (1):} Пусть $x \in Y \backslash \left(\bigcap \limits_{\alpha \in A} X_{\alpha} \right) ~ \Rightarrow$ $[x \in Y] ~ \& ~ \left[x \notin \bigcap \limits_{\alpha \in A} X_{\alpha} \right] \Rightarrow \exists ~ \alpha' \in A : ~ x \notin X_{\alpha'}$.\\

\noindent Пусть $x \in \bigcup \limits_{\alpha \in A} (Y \backslash X_{\alpha}) \Rightarrow \exists ~ \alpha' : ~ x \in Y \backslash X_{\alpha'} \Rightarrow [x \in Y] ~ \& ~ [x \notin X_{\alpha'}]$.\\

\noindent Таким образом в обоих случаях мы приходим к тому, что $\exists ~ \alpha' \in A : [x \in Y] ~ \& ~ [x \notin X_{\alpha'}]$.\\

\noindent \textbf{Доказательство (2):} Проводим аналогичные рассуждения.\\

\noindent \textbf{Теорема:} множество $X$ метрического пространства $\mathscr{M}$ является замкнутым $\Leftrightarrow$ $CX = \mathscr{M} \backslash X$ -- открытое множество. Причем $CX$ называется \textit{дополнением множества $X$}.\\

\noindent \textbf{Доказательство:} \textit{Необходимость} ($\Rightarrow$)\\

\noindent Доказываем от противного: предположим, что $CX$ не является открытым множеством $\Rightarrow$ $\exists ~ x_0 \in CX : ~ x_0$ не является внутренней точкой $CX$ $\Rightarrow$ 
\[\forall \varepsilon > 0 ~ \exists ~ x_{\varepsilon} \neq x_0 : ~ x_{\varepsilon} \notin CX ~ \& ~ x_{\varepsilon} \in B_{\varepsilon}(x_0)\]
тогда $x_0$ по определению является предельной точкой множества $X$ и при этом $x_0 \notin X$ (т. к. $x_0 \in CX$); получается, что существует предельная точка множества $X$, не лежащая в этом множестве, но по условию $X$ -- замкнутое множество, а значит сожержит все свои предельные точки, таким образом приходим к противоречи.\\

\noindent \textit{Достаточность} ($\Leftarrow$)\\

\noindent Доказываем от противного: предположим, что $X$ не является замкнутым множеством, тогда:

\[ \exists ~ x_0 \notin X : ~ x_0 \text{-- предельная точка множества} ~ X \]

\noindent По определению предельной точки:

\[ \forall \varepsilon > 0 ~ \exists ~ x_{\varepsilon} \neq x_0 : ~ x_{\varepsilon} \in X ~ \& ~ x_{\varepsilon} \in B_{\varepsilon}(x_0) \]

\noindent Тогда любой шарик $B_{\varepsilon}(x_0)$ с радиусом $r = \varepsilon$ и центром в точке $x_0$ не содержится в $CX ~ \Rightarrow$ $x_0$ не является внутренней точкой множества $CX$, но при этом $x_0 \in CX$, поскольку $x \notin X$; однако, поусловию $CX$ -- открытое множество, а значит должно содержать все свои внутренние точки. Таким образом, приходим к противоречию.\\

\noindent \textbf{Теорема:} замкнутые множества обладают следующими свойствами:

\begin{enumerate}
    \item $\mathscr{M}, \varnothing$ -- замкнутые множества.
    \item $\bigcap \limits_{\alpha \in A} X_{\alpha}$ -- пересечение любого числа замкнутых множеств $X_{\alpha}$ есть замкнутое множество.
    \item $\bigcup \limits_{j = 1}^K X_j$ -- объединение конечного числа замкнутых множеств $X_j$ есть замкнутое множество.
\end{enumerate}

\noindent \textbf{Доказательство свойства 2:} Воспользуемся теоремой о дополнении множества $X$: рассмотрим $C(\bigcap \limits_{\alpha \in A} X_{\alpha}) = \mathscr{M} \backslash (\bigcap \limits_{\alpha \in A} X_{\alpha}) = \bigcup \limits_{\alpha \in A} (\mathscr{M} \backslash X_{\alpha})$. $X_{\alpha}$ -- замкнутое множество $\Rightarrow$ $\mathscr{M} \backslash X_{\alpha}$ -- открытое множество $\Rightarrow$ $\bigcup \limits_{\alpha \in A} (\mathscr{M} \backslash X_{\alpha}) = C(\bigcap \limits_{\alpha \in A} X_{\alpha})$ -- открытое множество, тогда по теореме о дополнении множества $\bigcap \limits_{\alpha \in A} X_{\alpha}$ -- замкнутое множество.\\

\noindent \textbf{Доказательство свойства 3:} Воспользуемся теоремой о дополнении множества $X$: рассмотрим $C(\bigcup \limits_{j = 1}^K X_j) = \mathscr{M} \backslash (\bigcup \limits_{j = 1}^K X_j) = \bigcap \limits_{j=1}^K (\mathscr{M} \backslash X_j)$. $X_j$ -- замкнутое множество $\Rightarrow$ $\mathscr{M} \backslash X_j$ -- открытое множество $\Rightarrow$ $\bigcap \limits_{j=1}^K (\mathscr{M} \backslash X_j) = C(\bigcup \limits_{j = 1}^K X_j)$ -- открытое множество, тогда по теореме о дополнении множества $\bigcup \limits_{j = 1}^K X_j$ -- замкнутое множество.\\

\noindent \textbf{Замечание:}

\begin{enumerate}
    \item если $X$ -- открытое множество, то $X = int X$
    \item если $X$ -- замкнутое множество, то $\overline{X} = X$
    \item пусть $G$ -- открытое множество, тогда в общем случае $int(\overline{G}) \neq G$
    \item пусть $F$ -- замкнутое множество, тогда в общем случае $\overline{int F} \neq F$
\end{enumerate}

\subsection*{Внутренность, замыкание и граница множества.}

\noindent \textbf{Определение:} $int X$ -- совокупность всех внутренних точек множества $X \subset \mathscr{M}$ называется \textit{внутренностью множества X}.\\

\noindent \textbf{Определение:} $\overline X$ -- \textit{замыкание множества} $X \subset \mathscr{M}$ -- операция присоединения к множеству $X$ всех его предельных точек.\\

\noindent \textbf{Определение:} $\partial X$ -- граница множествва $X$ -- совокупность всех граничных точек множества $X$.\\

\subsection*{Компакты.}

\noindent \textbf{Определение:} Множество $X \subset \mathscr{M}$ называется \textit{компактом}, если если из любой последовательности его точек можно выделить сходящуюся подпоследовательность, предел которой принадлежит множеству $X$.

\[ \forall ~ \{x^n \} \subset X ~ \exists ~ \{x^{k_n} \} \subset X : \lim_{n \to \infty} x^{k_n} = a \in X \]

\noindent \textbf{Определение:} Множество $X \subset \mathbb{R}^m$, любые 2 точки которого можно соеденить лежащей в нем непрерывной кривой, называется \textit{линейно связным} (непрерывной кривой в m-мерной пространстве). (<<Введение в математический анализ.>> Л. Д. Кудрявцев том 2).\\

\noindent \textbf{Определение:} Множества $X_1$ и $X_2$ метрического пространства $\mathscr{M}$ называются \textit{отделимыми}, если ни одно из них не содержит точек рикосновения другого.\\

\noindent \textbf{Определение:} Множество $X$ метрического пространства $\mathscr{M}$ называется \textit{связным}, если его нельзя представить в виде объединения двух отделимых множеств.\\

\noindent \textbf{Определение:} Открытое связное множество называется \textit{областью}.

\subsection*{Метрическое пространство.}

\noindent \textbf{Определение:} Пусть $M$ -- произвольное множество, для любых точек $x, y \in M$ поставим в соответствие число $\rho(x, y) \geqslant 0$ такое что
\begin{enumerate}
    \item $\rho(x, y) = 0 ~ \Leftrightarrow ~ x = y$
    \item $\rho(x, y) = \rho(y, x)$
    \item $\rho(x, y) \leqslant \rho(x, z) + \rho(z, y)$
\end{enumerate}
\noindent тогда $\mathscr{M} = (M, \rho)$ называется \textit{метрическим пространством}, а функция $\rho(x, y)$ -- метрикой.\\

\noindent \textbf{Теорема [неравенство Коши-Буняковского]:} для любых $a_1, b_1, \dots, a_m, b_m$ справедливо:

\begin{equation*}
    \left( \sum_{j = 1}^{m} a_j b_j \right)^2 \leqslant \sum_{j = 1}^{m} a_j^2 \cdot \sum_{j = 1}^{m} b_j^2
\end{equation*}

\noindent \textbf{Доказательство:} Рассмотрим многочлен:
\[ p(z) = \sum_{j = 1}^{m} (a_j + b_j z)^2 = A + 2Bz + Cz^2 \]
\[ A = \sum_{j = 1}^{m} a_j^2; ~~ B = \sum_{j = 1}^{m} a_j b_j; ~~ C = \sum_{j = 1}^{m} b_j^2 \]
Заметим, что при любых значениях $z$ многочлен $p(z) \geqslant 0$, поскольку является суммой неотрицательных членов, тогда справедливо $B^2 - AC \leqslant 0 \Rightarrow B^2 \leqslant AC$ (дискриминант квадратного уравнения, деленный на 4). Подставляя $A, B, C$ получаем исходное неравенство.\\

\noindent \textbf{Теорема [неравенство Минковского]:} для любых $a_1, b_1, \dots, a_m, b_m$ справедливо:

\begin{equation*}
    \sqrt{\sum_{j = 1}^{m} (a_j + b_j)^2} \leqslant \sqrt{\sum_{j = 1}^{m} a_j^2} + \sqrt{\sum_{j = 1}^{m} b_j^2}
\end{equation*}

\noindent \textbf{Доказательство:} 

\[ \sum_{j = 1}^{m} (a_j + b_j)^2 = \sum_{j = 1}^{m} a_j^2 + 2\sum_{j = 1}^{m} a_j b_j + \sum_{j = 1}^{m} b_j^2 \]

\[ \sum_{j = 1}^{m} a_j^2 + 2\sum_{j = 1}^{m} a_j b_j + \sum_{j = 1}^{m} b_j^2 \stackrel{\text{К-Б}}{\leqslant} \left( \sqrt{\sum_{j = 1}^{m} a_j} \right)^2 + 2\sqrt{\sum_{j = 1}^{m} a_j} \cdot \sqrt{\sum_{j = 1}^{m} b_j} + \left( \sqrt{\sum_{j = 1}^{m} b_j} \right)^2 \]

\noindent Свернем правую чать по формуле квадрата суммы и получим:

\[ \sum_{j = 1}^{m} (a_j + b_j)^2 \leqslant \left( \sqrt{\sum_{j = 1}^{m} a_j^2} + \sqrt{\sum_{j = 1}^{m} b_j^2} \right)^2 \]

\noindent \textbf{Примеры метрических пространств:}

\begin{equation*}
    \mathscr{M} = (M, \rho), ~ \rho = 
    \begin{cases}
        0, x = y\\
        1, x \neq y
    \end{cases}
\end{equation*}

\begin{equation*}
    \mathbb{E}^m = (\mathbb{R}^m, \rho_e), ~ \rho_e(x, y) = \sqrt{\sum_{j = 1}^m (x_j - y_j)^2}
\end{equation*}
    
\begin{equation*}
    \mathscr{M} = (\mathbb{R}^m, \rho_{1}), ~ \rho_{1}(x, y) = \underset{1 \leqslant j \leqslant m}{max} |x_j - y_j|
\end{equation*}

\subsection*{Компакты в метрическом пространстве и описание компактов в n–мерном евклидовом пространстве.}

\noindent \textbf{Определение:} Множество $X \subset \mathbb{E}^m$ называется ограниченным, если существует m-мерный шар $B_{R}(O)$ такой, что $X \subset B_{R}(O)$ (<<Курс математического анализа>> Л. Д. Кудрявцев том 2).\\

\noindent \textbf{Теорема:} $X \subset \mathbb{E}^m$ является компактом $\Leftrightarrow$ $X$ -- ограниченное и замкнутое множество.\\

\noindent \textbf{Доказательство:} \textit{Необходимость} ($\Rightarrow$)\\
\noindent Пусть $X \subset \mathbb{E}^m$ является компактом, докажем, что $X$ является замкнутым множеством. Возьмем произвольную предельную точку $a$ множества $X$ и будем рассматривать ее окрестности: $B_{\frac{1}{n}}(a)$:
\begin{center}
    $r_1 = 1$, тогда по определению предельной точки $\exists ~ x_1 \neq a : x_1 \in X ~ \& ~ x_1 \in B_{1}(a)$\\
    $r_2 = \frac{1}{2}$, тогда по определению предельной точки $\exists ~ x_2 \neq a : x_2 \in X ~ \& ~ x_2 \in B_{2}(a)$\\
    $\dots$\\
    $r_n = \frac{1}{n}$, тогда по определению предельной точки $\exists ~ x_n \neq a : x_n \in X ~ \& ~ x_n \in B_{n}(a)$\\
    $\dots$\\
\end{center}
\noindent таким образом, мы построили последовательность точек $\{x^n \} \subset X$ такую, что выполняется следующее:
\[ \forall \varepsilon > 0 ~ \exists N = \frac{1}{\varepsilon} : ~ \forall n \geqslant N \mapsto \rho(a, x^n) < \frac{1}{n} < \varepsilon \]
или, что то же самое:
\[\lim_{n \to \infty} x^n = a \]
но по условию, $X$ -- компакт, а значит $a \in X$, таким образом, в силу произвольности точки $a$, компакт $X$ содержит все свои предельные точки, а значит, является замкнутым множеством.\\

\noindent Заметим, что неограниченное множество $X$ не может быть компактом, так как в неограниченном множестве можно построить последовательность точек, которая не будет являться сходящейся.\\

\noindent \textit{Достаточность} ($\Leftarrow$)\\
\noindent Пусть $X \subset \mathbb{E}^m$ -- ограниченное, замкнутое множество. Возмем последовательность точек $\{x^n \} \subset X$, по теореме Больцано-Вейерштраса, в силу ограниченности этой последовательности, из нее можно выделить сходящуюся подпоследовательность $\{x^{k_n} \} \xrightarrow[n \to \infty]{} a$, в силу замкнутости множества $a \in X$, но тогда получается, что $X$ -- компакт.

\end{document}