\documentclass[a4paper,12pt]{article} % тип документа

%  Русский язык
\usepackage[T2A]{fontenc}			% кодировка
\usepackage[utf8]{inputenc}			% кодировка исходного текста
\usepackage[english,russian]{babel}	% локализация и переносы

\usepackage{graphicx}               % импорт изображений
\usepackage{wrapfig}                % обтекаемые изображения
\graphicspath{{pictures/}}          % обращение к подкаталогу с изображениями
\usepackage[14pt]{extsizes}         % для того чтобы задать нестандартный 14-ый размер шрифта
\usepackage{amsfonts}               % буквы с двойными штрихами
\usepackage[warn]{mathtext}         % русский язык в формулах
\usepackage{indentfirst}            % indent first
\usepackage[margin = 25mm]{geometry}% отступы полей
\usepackage{amsmath}                % можно выводить фигурные скобочки -- делать системы уравнений
\usepackage[table,xcdraw]{xcolor}   % таблицы
\usepackage{amsmath,amsfonts,amssymb,amsthm,mathtools} % Математика
\usepackage{wasysym}                % ???
\usepackage{upgreek}                % ???  

\usepackage{gensymb} % degree symbol
\usepackage{mathrsfs}

%Заговолок
%\author{Паншин Артём и Белов Владислав}
%\title{Билет 7.}
%\date{\today}


\begin{document} % начало документа

%\maketitle
\newpage

\section*{Билет 7.}

\subsection*{Некоторые свойства определенного интеграла:}

\begin{enumerate}
    \textbf{Свойство 1}\\[5mm] $\int \limits_a^a f(x)dx = 0$\\[2mm]
    \textbf{Свойство 2}\\[5mm] $\int \limits_a^b f(x)dx = -\int \limits_b^a f(x)dx$\\[2mm]
    \textbf{Свойство 3}\\[5mm] Если $ f $, $ g $ интегрируемы на $ [a, b] $, то $\forall{\alpha, \beta} \in \mathbb{R} $ функция $ h = \alpha f + \beta g $ интегрируема на $ [a, b] $. \\ [5 mm]
    \textbf{Доказательство:} \\[5 mm]
    $ I_n \{ \tau, \xi \} = \sum\limits_{j = 1}^n   [\alpha f(\xi_j) + \beta g(\xi_j)] \Delta x_j  =  \alpha \cdot \sum\limits_{j = 1}^n f(\xi_j)\Delta x_j + \beta \cdot \sum\limits_{j = 1}^n g(\xi_j)\Delta x_j = $ \\ [2 mm] $ \alpha I_f \{\tau, \xi \} + \beta I_g \{\tau, \xi \} $.   \\ [2 mm] 
    \textbf{Свойство 4}\\[5mm] Если $ f $ и $ g $ интегрируемы на $ [a, b] $, то $ h = f \cdot g $ интегрируема на $ [a, b] $. \\ [2mm]
    \textbf{Доказательство:} \\[3 mm]
    $ \exists A > 0 \wedge  \exists B > 0: |f(x)| \leq A, \hspace*{1mm} |g(x)| \leq B \hspace*{2mm} \forall x \in [a, b]$ $\Rightarrow$ $ h $ ограничена на $ [a, b]$ \\ [2 mm]
    $| h(x') - h(x'') | = | f(x')g(x') - f(x'')g(x'')| = |f(x')g(x') - f(x'')g(x') + $ \\ [2 mm] $|f(x'')g(x') - f(x'')g(x'') | \leq |g(x')| \cdot |f(x') - f(x'')| + |f(x'')| \cdot | g(x') - $ \\ [2 mm] $ g(x'') | \leq B|f(x') - f(x'')| + A|g(x') - g(x'')| $ $\Rightarrow$  $[ M_j(h) - m_j(h)] \leq $ \\ [2mm] $  B[M_j(f) - m_j(f)] + A[M_j(g) - m_j(g)] $ \\ [3 mm]
    $ f, g $ интегрируемы на $ [a, b] $ $ \Rightarrow $ $ \forall \varepsilon > 0 \hspace*{2mm} \exists T' : \overline{S}_{T'}(f) - \underline{S}_{T'} (f) < \frac {\varepsilon}{2B} \\ [2 mm] \hspace*{83mm} \exists T'' : \overline{S}_{T''}(g) - \underline{S}_{T''} (g) < \frac {\varepsilon}{2A} $ \\ [3 mm]
    $ T = T' \cup T''$ \\ [2 mm]
    $ \underline{S}_{T'}(f) \leq \underline{S}_{T}(f) \leq \overline{S}_{T}(f) \leq \overline{S}_{T'}(f) $ $ \Rightarrow $ $\overline{S}_{T}(f) - \underline{S}_{T} (f)  \leq \overline{S}_{T'}(f) - $ \\ [2mm] $\underline{S}_{T'} (f) < \frac {\varepsilon}{2B} $ \\ [2 mm]
    $ \underline{S}_{T''}(g) \leq \underline{S}_{T}(g) \leq \overline{S}_{T}(g) \leq \overline{S}_{T''}(f) $ $ \Rightarrow $ $\overline{S}_{T}(g) - \underline{S}_{T} (g)  \leq \overline{S}_{T''}(g) - $ \\ [2mm] $\underline{S}_{T''} (g) < \frac {\varepsilon}{2A} $ $ \Rightarrow $ \\ [3 mm]
    $\overline{S}_{T}(h) - \underline{S}_{T} (h) < A \cdot \frac {\varepsilon}{2A} + B \cdot \frac{\varepsilon}{2B} = \varepsilon$ $\Rightarrow$ $ h $ интегрируемая на $ [a, b] $.
    
    \textbf{Свойство 5}\\[5mm] $f$ интегрируема на $[a,b] \And [c,d] \in [a,b]\Rightarrow f$ интегрируема на $[c,d]$\\ [5mm]
    \textbf{Доказательство:}\\[2mm]
    $\forall \varepsilon > 0 \exists T: \overline{S_T}- \underline{S_T}< \varepsilon$\\[2mm]
    $T' = T \cup \{c,d\}$\\[2mm]
    $\overline{S_{T'}}-\underline{S_{T'}}\leq \overline{S_T}-\underline{S_T} < \varepsilon$\\[2mm]
    $T^*$ порожденное разбиением $T' \Rightarrow \overline{S_{T^*}}- \underline{S_{T^*}}\leq  \overline{S_T}-\underline{S_T} < \varepsilon$\\[2mm]
    
    \textbf{Свойство 6}\\[5mm] Если $ f $ интегрируема на отрезке $ [a, c]  $ и $ [c, b]  $, то $ f $ интегрируема на $ [a, b]  $ и \vspace*{1mm} \hspace*{50mm} $$\int\limits_a^b f(x)dx = \int\limits_a^c f(x)dx + \int\limits_c^b f(x)dx $$ \newline
    \textbf{Доказательство:} \\[3 mm]
    \hspace*{5mm} Пусть $ a < c < b$: \\[2 mm]

    $ \forall \varepsilon > 0 \hspace*{2mm}  \exists T^{'}, T^{''} $ отрезков $ [a, c]  $ и $ [c, b]  $ \hspace*{2mm}  $\overline {S_{T^'}}$ $ - {\underline{S}_{T^'}}  <  {\frac{\varepsilon}{2}} $, \hspace*{2mm}  \vspace*{1mm}
    $\overline {S}_{T^{''}}$ $ - {\underline{S}_{T^{''}}}  <  {\frac{\varepsilon}{2}} $ \newline
    $T = T' \cup T{''}$ --- разбиение отрезка $ [a,b] $. \\[2mm]
    $ \underline{S}_{T'} =  \underline{S}_{T}^1 \leq \overline{S}_{T}^1 = \overline{S}_{T'} $ \\ [2mm]
    $ \underline{S}_{T''} =  \underline{S}_{T}^2 \leq \overline{S}_{T}^2 = \overline{S}_{T''} $ \\ [2mm]
    $ \overline{S}_{T} -  \underline{S}_{T} = \overline{S}_{T}^1 + \overline{S}_{T}^2 - \underline{S}_{T}^1 + \underline{S}_{T}^2 < \varepsilon $ $ \Rightarrow f $ интегируема на $ [a, b]$ $ \Rightarrow $ интегральная сумма на $ [a, b] $ есть сумма интегральных сумм на $ [a,c] $ и $ [c, b] $ \\ [2mm]
    \hspace*{5mm} Пусть $ c < a < b $ или $ a < b < c $: \\[2 mm]
    $ [a, b] $ есть часть отрезка $ [c, b] $ или $ [a, c] $ $ \Rightarrow $ ввиду того, что интегрируемая на отрезке интегрируема на любом его участке, то $ f $ интегрируема на $ [a, b] $. \\ [2 mm]
    \hspace*{5mm} Пусть $ a < b < c $ : \\[2 mm]
    $\int\limits_a^b f(x)dx + \int\limits_b^c f(x)dx = \int\limits_a^c f(x)dx  \newline
    \int\limits_a^b f(x)dx = \int\limits_a^c f(x)dx - \int\limits_b^c f(x)dx  \Rightarrow \int\limits_a^b f(x)dx = \int\limits_a^c f(x)dx + \int\limits_c^b f(x)dx $ \\ [2mm]
    Аналогично доказывается для $ c < a < b$. \\ [2mm]
    
   \textbf{Свойство 7}\\[5mm] Пусть $f$ ограничена на $(a,b], \forall \alpha > 0: ~ 0<\alpha<b-a, f $ интегрируема на $[\alpha+a, b]$,тогда при любом доопределении $f$ в точке $a$, получится функция, интегрируемая на $[a,b]$ и $\int\limits_a^b f(x) dx = \lim \limits_{\alpha \rightarrow 0} \int\limits_{a+\alpha}^b f(x)dx$\\[2mm] 
   \textbf{Доказательство:}\\[2mm]
    $\exists A>0: \forall x \in (a,b] \longmapsto |f(x)|\leq A, f(a) = B$\\[2mm]
    $M = max\{A, |B|\} \Rightarrow \forall x \in [a,b], |f(x)|\leq M$\\[2mm]
    $\forall\varepsilon > 0 \exists \alpha = \alpha(\varepsilon):2M\alpha< \varepsilon/2$\\ [2mm]
    Для $[a+\alpha, b]$ найдется такое  $\exists T: \overline{S_T}- \underline{S_T}< \varepsilon/2$\\[2mm]
    $\exists T' = T \cup \{a\}, \overline{S_{T'}}- \underline{S_{T'}}= \overline{S_T}- \underline{S_T}+(M_0-m_0)\alpha<\varepsilon/2+2M\alpha<\varepsilon/2+\varepsilon/2=\varepsilon $\\[5mm]
\subsection*{Оценки определенного интеграла:}   
\textbf{Оценка 1:}\\[2mm]
    $f$  интегрируема на $[a,b] \And f(x)\geq 0\forall x \in [a,b]\Rightarrow \int\limits_a^b f(x)dx$\\[2mm]
    \textbf{Доказательство:}\\[2mm]
    $f(x)\geq 0 \forall x \in [a,b]\Rightarrow \forall T \And \forall \{\upxi\}\mapsto I\{T,\upxi \}\geq 0, I ~-~$ предел интегральных сумм\\[2mm]
    Теперь надо доказать, что при $\Delta_T \rightarrow 0 \mapsto I\geq 0$ \\[2mm]
    От противного:\\[2mm] 
    $I<0 \Rightarrow \varepsilon = \frac{|I|}{2} \exists\delta(\varepsilon)>0: \forall T, \Delta_T< \delta \mapsto |I\{T,\upxi\}-I|< \frac{|I|}{2}\Rightarrow I-\frac{|I|}{2}<I\{T,\upxi\}<I+\frac{|I|}{2}<0\Rightarrow I\{T,\upxi\}<0 ~-~ $ противоречие.\\[5mm]
    \textbf{Оценка 2:}\\[2mm]
    $f$ непрерывна на $[a,b] \And f(x) \geq 0 \forallx \in [a,b] \And f\not\equiv 0 \Rightarrow \int\limits_a^b f(x)dx \geq j>0$\\[2mm]
    \textbf{Доказательство:}\\[2mm]
    $\exists x_0 \in (a,b): f(x_0) = 2\alpha>0 \Rightarrow$ [по теореме о сохранении знака непрерывной функции] $\Rightarrow \exists [c,d] \subset [a,b], x  \in [c,d]: f(x)\geq \alpha > 0 $ на $[c,d]\Rightarrow f(x)-\alpha \geq 0$ на $[c,d]\stackrel{\text{св-во 5 и оц-ка 1}}{\Rightarrow}\int \limits_c^d (f(x)-\alpha) dx\geq 0 \Rightarrow \int \limits_c^d f(x)dx\geq \int\limits_c^d\alpha dx = \alpha(d-c) = j > 0$\\[2mm]
    $\int \limits_c^d f(x) dx \geq j > 0 \Rightarrow \int \limits_a^c  f(x) dx  + \int \limits_c^d  f(x) dx+ \int \limits_d^b  f(x) dx \geq 0+j+0>0$\\[2mm]
    \textbf{Оценка 3:}\\[2mm]
    $f,g$ интегрируемы на $[a,b] \And \forall x \in [a,b]\mapsto f(x)\geq g(x) \Rightarrow\int \limits_a^b f(x) dx \geq \int \limits_a^b g(x) dx$\\[2mm]
    \textbf{Доказательство:}\\[2mm]
    $f(x)-g(x) \geq 0 \forall x \in [a,b] \stackrel{\text{оц-ка 1}}{\Rightarrow} \int \limits_a^b [f(x)-g(x)]dx\geq 0 \stackrel{\text{св-во 2}}{\Rightarrow}\int \limits_a^b f(x)dx - \int \limits_a^b g(x) \geq 0 $\\[2mm]
    
    \textbf{Оценка 4:}\\[2mm] Если $ y = f (x) $ интегрируема на $ [a,b] $, то $ y = |f(x)| $ интегрируема на $ [a, b]$ и \vspace*{1mm} \hspace*{50mm} $$\bigg|\int\limits_a^b f(x)dx\bigg| \leq \int\limits_a^b |f(x)|dx $$  \\ [2mm]
    \textbf{Доказательство:} \\ [3 mm]
    \hspace*{5mm} Пусть $ |f| $ --- интегрируема. \\ [2mm]
    $ T = \{ a = x_0 < x_1 < \dots < x_n = b \}$ \\ [2 mm]
    $ M_j = \sup\limits_{[x_{j - 1}, x_j]} f(x)$, $ \hspace*{50mm} $ $m_j =  \inf\limits_{[x_{j - 1}, x_j]} f(x) $ \\ [2 mm] 
     $ \overline{M}_j = \sup\limits_{[x_{j - 1}, x_j]} |f(x)|$, $ \hspace*{50mm} $  $ \overline{m}_j =  \inf\limits_{[x_{j - 1}, x_j]} |f(x)| $ \\ [3mm]
     $ \overline{M}_j - \overline{m}_j \leq M_j - m_j \hspace*{5mm} ( \textasteriskcentered )$ 
     \begin{enumerate}
     \item[1)] $M_j > 0, \hspace*{5} m_j > 0 $ $\Rightarrow$ очевидное равенство в $ (\textasteriskcentered) $
     \item[2)] $M_j < 0, \hspace*{5} m_j < 0 $ $\Rightarrow$ очевидное равенство в $ (\textasteriskcentered) $
     \item[3)] $ M_j > 0, m_j < 0 $ $\Rightarrow$ $ \overline{M}_j - \overline{m}_j < M_j - m_j $
     \end{enumerate} \\ [2 mm]
     Из $ (\textasteriskcentered) $ следует: \\ [2 mm]
     $ \overline{S}_T(|f|) -\underline{S}_T(|f|) \leq \overline{S}_T(f) -\underline{S}_T(f) < \varepsilon$ \\ [2 mm]
     $ \forall \varepsilon > 0 \hspace{2 mm} \exists T: \hspace{2 mm} \overline{S}_T(|f|) -\underline{S}_T(|f|) < \varepsilon$ $\Rightarrow$ $ |f| $ интегрируема и \\[2mm] $ - |f(x)| $  $\leq $ $f (x)$ $\leq$ $|f (x)| $ \\ [2mm]
     Вспомним, что если $ y = f(x) $ и $ y = g(x) $ интегрируемы на $ [a,b] $ и $ f(x) \geq g(x) \hspace{1 mm} \forall x \in [a, b]$, то $ \int\limits_a^b f(x)dx \geq \int\limits_a^b g(x)dx $, тогда \\ [1 mm]
     $$ - \int\limits_a^b |f(x)| dx \leq \int\limits_a^b f(x) dx \leq \int\limits_a^b |f(x)| dx $$ $\Rightarrow$ \\
     $$ \bigg|\int\limits_a^b f(x)dx\bigg| \leq \int\limits_a^b |f(x)|dx $$ \\ 
     \textbf{Замечание:} \\
     $ |f| $ --- интегрируема $\not\Rightarrow$ $ f $  --- интегрируема \\
     \textbf{Пример:} \\
     \begin{equation*}
    y = \tilde D(x) = \begin{cases}
        \hspace{4 mm}1, \hspace{3 mm} x \in \mathbb{Q};\\
        -1,  \hspace{3 mm} x \in \hspace{1 mm} \mathbb{I};
    \end{cases}
     \end{equation*} \\ [2 mm]
     
    \textbf{Оценка 5:}\\[2mm] Пусть $ y = f(x) $, $ y = g(x) $ интегрир. на $ [a, b] $ и $ g(x) \geq 0 $ $\forall x \in [a, b] $. \\ [2 mm]
    Если $ M = \sup\limits_{[a, b]} f (x), m = \inf\limits_{[a, b]} f(x) $, то \\
    $$ m\int\limits_a^b g(x)dx \leq \int\limits_a^b f(x) \cdot g(x)dx \leq M \int\limits_a^b g(x)dx  $$ \\ [2 mm]
    \textbf{Доказательство:} \\ [2mm]
    $ m \leq f (x) \leq M \hspace{2 mm} \forall x \in [a, b]  \hspace{2 mm} g (x) \geq 0 $ $ \Rightarrow $
    $ m \cdot g(x) \leq f(x) \cdot g(x) \leq M \cdot g (x) $ \\ [2 mm] $ \Rightarrow $ исходное условие доказано исходя из оценки 3 и свойства 3 \\ [2 mm]
    
\end{enumerate}
\textbf{Предложение [Формула среднего значения]: } \\ [2 mm]
Пусть $ f $ интегрируема на $ [a, b] $, $ M = \sup\limits_{[a, b]} f(x), m = \inf\limits_{[a, b]} f(x) $. Тогда $ \exists \mu: \hspace{2 mm} m \leq \mu \leq M $ такое, что \\
$$ \int\limits_a^b f(x)dx = \mu (b - a)$$ \\
\textbf{Доказательство:} \\ [2 mm]
Из оценки интегрирования неравенств (результата предыдущего пункта) при $ g \equiv 1 $ $\Rightarrow$ $ m (b - a) \leq \int\limits_a^b f (x)dx \leq M(b - a) $ $ \Rightarrow $ $\displaystyle \mu = \frac {\int\limits_a^b f(x)dx}{b -a }$ \\ [2 mm]
\textbf{Теорема [Интегральная теорема о среднем]:} \\ [2mm]
Пусть $ f $ и $ g $ интегрируемы на $ [a, b] $. $ M = \sup\limits_{[a,b]} f(x), m = \inf\limits_{[a,b]} f(x) $ и \\ [2 mm] $ g(x) \geq 0 \hspace{2mm} \forall x \in [a, b] $ (либо  $g (x) \leq 0 $). Тогда $\exists \mu: m \leq \mu \leq M $ такая, что \\
$$ \int\limits_a^b f(x) \cdot g(x)dx = \mu \int\limits_a^b g(x)dx$$ \\
В частности, если $ f $ непрерывна на $ [a, b] $, то $\exists \hspace{1 mm} \xi \in [a,b]: $ \\ 
$$ \int\limits_a^b f(x) \cdot g(x)dx = f(\xi) \cdot \int\limits_a^b g(x)dx$$ \\ [6 mm]
\textbf{Доказательство:} \\ [2mm]
\begin{enumerate}
     1) $\int\limits_a^b g(x)dx = 0 $ \\
     $\Rightarrow$ оценка интегрирования неравенства $\Rightarrow$ $\int\limits_a^b f(x)g(x) = 0 $ и $\mu $ --- любое число\\  
     
     
     2) $\int\limits_a^b g(x)dx > 0 $ \\
     $\Rightarrow$ Оценка интегрирования неравенства \\ [2mm]
     $\displaystyle m \leq \frac{\int\limits_a^b f(x)dx}{\int\limits_a^b g(x)dx} \leq M$ и $\displaystyle \mu = \frac{\int\limits_a^b f(x)dx}{\int\limits_a^b g(x)dx} $ \\
     Если $ f $ непрерывна на $ [a,b] $ $\Rightarrow$ $\exists \hspace{1 mm} \xi: \mu = f(\xi) $
     \end{enumerate} \\ [2 mm]
     \textbf{Предложение:}\\[2mm]
     Пусть $f$ интегрируема на $[a,b], m = \inf\limits_{[a,b]}f, M = \sup \limits_{[a,b]}f \Rightarrow \exists \mu : m\leq \mu \leq M: \int \limits_a^b f(x)g(x)dx = \mu(b-a),$ если $f$ непрерывна на $[a,b]$, то $\exists \upxi \in [a,b]: \int \limits_a^b f(x)dx = f(\upxi)(b-a)$  $[g\equiv 1]$


\subsection*{Интегралы с переменным верхним пределом. Вычисление определеннных интегралов}
\textbf{Определение:} \\ [2mm]
Пусть $ y = f(x) $ интегрируема на $ [a,b] $ $\Rightarrow$ $\hspace{2mm}$ $ \forall x \in [a,b] $ существует \\
$$ \int\limits_a^x f(t)dt = F(x) $$ \\
Этот интеграл называется интегралом с переменным верхним пределом \\ [2 mm]
\textbf{Теорема:} \\ [2mm]
Любая непрерывная на $ [a,b] $ функция $ y = f(x) $ имеет на этом отрезке первообразную. Одной из первообразных является функция \\
$$ F(x) = \int\limits_a^x f(t)dt, x \in [a,b] $$ \\ [2 mm]
\textbf{Доказательство:} \\ [2 mm]
$ \forall x \in [a, b], \hspace{1mm} x + \Delta x \in [a, b] $. Докажем, что $$ \lim_{\Delta x  \rightarrow 0} \frac{F(x + \Delta x) - F(x) }{\Delta x} = f (x)$$ \\ [2 mm]
$\displaystyle F(x + \Delta x) - F(x) = \int\limits_x^{x + \Delta x} f(t)dt$ \\ [2 mm]
По теореме о среднем $\exists \xi$, лежащая между $ x $ и $ x + \Delta x:$ $ F(x + \Delta x ) - F(x) = f(\xi) \Delta x$ $\Rightarrow$ $\displaystyle \frac{F(x + \Delta x) - F(x)}{\Delta x} = f(\xi)$ \\ [2mm]
Так как $ f $ непрерывна на $ [a,b] $, то при $ \Delta x \rightarrow 0$ $\Rightarrow$ $f (\xi) \xrightarrow[\Delta x \rightarrow 0]{} f (x)$ и \\ [2mm]

$$ \lim_{\Delta x  \rightarrow 0} \frac{F(x + \Delta x) - F(x) }{\Delta x} = F' (x)$$ 
\textbf{Замечание:}\\ [2mm]
Из доказательства теоремы следует, что
$$\frac{d}{dx} \int\limits_a^x f(t)dt = f(x)$$\\ [2mm]
\textbf{Предложение:} \\ [2mm]
Если $f$ интегрируема на $[a,b]$, то $F$ непрерывна на $[a,b]$\\ [2mm]
\textbf{Доказательство:} \\[2mm]
$\forall \in [a,b], x+\Delta x\in [a,b]$ \\ [2mm]
$F (x+ \Delta x ) - F(x) = \Delta F(x+\Delta x)$ \\[2mm]
$\Delta F(x, \Delta x) = \int\limits_x^{x + \Delta x} f(t)dt = \mu \Delta x : m \leq \mu \leq M $ (Формула среднего значения) \\[2mm]
$\Delta x \rightarrow 0 \Rightarrow \Delta F(x, \Delta x) \rightarrow 0 \Rightarrow F$ непрерывна в $X$ \\[2mm]
\textbf{Замечание:}\\[2mm]
Если $f$ непрерывна на $[a,b] \Rightarrow \forall \upphi(x) =\int\limits_a^{x} f(t)dt +C $\\ [2mm]
$\upphi(a) = C,~ \upphi(b)= \int\limits_a^{b} f(x)dx + C ~~\Rightarrow\int\limits_a^b f(x)dx = \upphi(b)-  \upphi(a) $\\ [2mm]
\textbf{Теорема [Формула Ньютона-Лейбница]:}\\ [2mm]
Если $f$ непрерывна на $[a,b]$, то $\int\limits_a^{b} f(x)dx = \upphi(b)-  \upphi(a) $, где $\upphi ~-~$ любая перавообразная функции $f$\\ [2mm]
\textbf{Доказательство:} \\[2mm]
См. предыдущее замечание.\\[2mm]
\textbf{Теорема 7':}\\[2mm]
Если $f$:\\[2mm]
1) интегрируема на $[a,b]$;\\[2mm]
2) обладает на $[a,b]$ первообразной $\upphi $;\\[2mm]
то справедлива формула $\int\limits_a^{b} f(x)dx = \upphi(b)-  \upphi(a) $\\[2mm]
\textbf{Замечание:}\\[2mm]
1) $y = sgn x, x \in [-1,1]$ интегрируема на $[-1,1]$, но не обладает первообразной.\\[2mm]
2) $F(x)= \begin {cases} 2x \sin \frac{1}{x^2}, |x|\leq 1, x \neq 0\\ 0, x = 0
\end{cases}$ является первообразной для \\[2mm]
$f(x) = \begin{cases} 2x \sin\frac{1}{x^2} - \frac{2}{x} \cos\frac{1}{x^2}, |x|\leq 1, x \neq 0\\ 0, x =0
\end{cases}$\\ [2mm]
$F'(0) =  \lim\limits_{x\rightarrow 0}\frac{x^2 sin\frac{1}{x^2}}{x} = 0$ $f$ не является интегрируемой на $[-1,1]$  (не ограничена)\\[2mm]
\textbf{Теорема [Замена переменных в определенном интегрировании]:}\\[2mm]
Пусть выполнены следующие условия:\\[2mm]
1) $y = f(x)$ непрерывна на $[a,b]$\\[2mm]
2) $x = g(t)$ непрерывно дифференцируема на $[a,b]$\\[2mm]
3) $g(\alpha) = a, g(\beta) = b$ и $\forall t \in [\alpha, \beta] \longmapsto a \leq g(t)\leq b $\\ [2mm]
тогда справедлива формула $\int\limits_a^b f(x) dx = \int \limits_\alpha^\beta f(g(t))g'(t) dt$\\[2mm]
\textbf{Доказательство:}\\[2mm]
$\upphi ~-~$ первообразная функции $f\Rightarrow \int\limits_a^b f(x)dx = \upphi(b)- \upphi(a)$.\\[2mm]
Т.к. $\upphi$ и $g$ дифференцируемы на $[a,b]$ и $[\alpha, \beta]$ соответственно, то\\[2mm] 
$\frac{d}{dt}\Big[\upphi(g(t))\Big]= \upphi'(g(t))\cdot g'(t),$ но $\upphi'(x) = f(x) \rightarrow \upphi'(g(t))= f(g(t))\Rightarrow$\\ [2mm]
$\frac{d}{dt}\Big[\upphi(g(t))\Big]=f(g(t))\cdot g'(t)$\\ [2mm]
По условию $f(g(t))\cdot g'(t)$ непрерывна на $[\alpha, 
\beta]$ и $\upphi(g(t))~-~$ её первообразная.\\[2mm]
$$\int_\alpha^\beta f(g(t))g'(t)dt = \upphi(g(\beta))- \upphi(g(\alpha))= \upphi(b) - \upphi(a) = \int_a^b f(x)dx $$
\textbf{Теорема [Формула интегрирования по частям]:}\\[2mm]
Пусть $u = u(x), v= v(x)$ непрерывно дифференцируемые на $[a,b]$. Тогда $$\int_a^b udv = [uv]|^b_a - \int_a^b vdu$$
\textbf{Доказательство:}\\[2mm]
Функция $u \cdot v$ является первообразной функции $uv'+u'v$. Каждая их этих функций непрерывная $\Rightarrow$
$$\int_a^b[uv'+u'v]dx = [uv]|^b_a$$

\end{document}
