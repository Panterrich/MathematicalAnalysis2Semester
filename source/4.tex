\documentclass[a4paper,12pt]{article} % тип документа

%  Русский язык
\usepackage[T2A]{fontenc}			% кодировка
\usepackage[utf8]{inputenc}			% кодировка исходного текста
\usepackage[english,russian]{babel}	% локализация и переносы

\usepackage{graphicx}               % импорт изображений
\usepackage{wrapfig}                % обтекаемые изображения
\graphicspath{{pictures/}}          % обращение к подкаталогу с изображениями
\usepackage[14pt]{extsizes}         % для того чтобы задать нестандартный 14-ый размер шрифта
\usepackage[warn]{mathtext}         % русский язык в формулах
\usepackage{indentfirst}            % indent first
\usepackage[margin = 25mm]{geometry}% отступы полей
\usepackage[table,xcdraw]{xcolor}   % таблицы
\usepackage{amsmath,amsfonts,amssymb,amsthm,mathtools} % Математика
\usepackage{wasysym}                % ???
\usepackage{upgreek}                % ???  
\usepackage{caption}
\captionsetup{labelsep=period}
\usepackage{mathrsfs}
\usepackage{makecell}
\usepackage{gensymb} % degree symbol



\pagestyle{empty}

\begin{document}
	
\section*{Билет номер 4}

	\subsection*{Частные производные высших порядков}
	
	\textbf{Определение:} Пусть $\omega = f(x)$ - дифференцируема в $D \subset \mathbb{E}^m$, $D$ - область.
	И $\forall x \in D \text{ }\exists \frac{\partial{f}}{\partial{x_j}}, j = \overline{1, m}$.
	
	Пусть $g_j = \frac{\partial{f}}{\partial{x_j}}$, и в точке $x$ $\exists\frac{\partial{g_j}}{\partial{x_k}}$. Тогда
	\[
		\frac{\partial{g_j}}{\partial{x_k}} = \frac{\partial{ }}{\partial{x_k}}\left(\frac{\partial{f}}{\partial{x_k}}\right)
	\]
	называется частной производной 2-го порядка функции $f$ в точке $x$. Частные производные высших порядков определяются так же.
	
	\textbf{Обозначения:} 
	\[		
		\frac{\partial^2f}{\partial{x_k}\partial{x_j}}(x), \text{   } f^{''}_{x_jx_k}(x), \text{   } f^{(2)}_{x_jx_k}(x)
	\]
	\[
		j = k: \text{ }\frac{\partial^2f}{\partial{x_j}^2}(x)
	\]
	
	\textbf{Примечание:} если $k\neq j$, производная $\frac{\partial^2f}{\partial x_k\partial x_j}$ называется смешанной.
	
	\subsection*{Независимость смешанной частной производной от порядка дифференцирования}
	
	\textbf{Примеры:}
	\begin{align*}
		1) f(x, y) &= \text{arctg}\left(\frac{x}{y}\right)\\
		\frac{\partial{f}}{\partial{x}} &= \frac{1}{1+\frac{x^2}{y^2}}\cdot\frac{1}{y} = \frac{y}{y^2+x^2}\\
		\frac{\partial{f}}{\partial{y}} &= \frac{1}{1+\frac{x^2}{y^2}}\cdot\left(-\frac{x}{y^2}\right) = -\frac{x}{y^2+x^2}\\
		\frac{\partial^2f}{\partial{y}\partial{x}} &= \frac{y^2+x^2-2y^2}{(x^2+y^2)^2} = \frac{x^2-y^2}{(x^2+y^2)^2}\\
		\frac{\partial^2f}{\partial{x}\partial{y}} &= -\frac{y^2+x^2-2x^2}{(x^2+y^2)^2} = \frac{x^2-y^2}{(x^2+y^2)^2}
	\end{align*}
	\[
		\frac{\partial^2f}{\partial{x}\partial{y}} = \frac{\partial^2f}{\partial{y}\partial{x}}
	\]
	
	\vspace{10mm}
	$$
	2)f(x, y) = 
	\begin{cases*} 
		xy\cdot\frac{x^2-y^2}{x^2+y^2},& \text{$x^2+y^2\neq 0$}\\
		0,& \text{$x^2+y^2= 0$}
	\end{cases*}$$
    \begin{align*}
	f(x, 0) &= f(0, y) = f(0, 0) = 0 \Rightarrow \frac{\partial f}{\partial x}(0, 0) = \frac{\partial f}{\partial y}(0, 0) = 0\\&\\
	f^{'}_{x} &= y\frac{x^2-y^2}{x^2+y^2}+xy\frac{2x(x^2+y^2) - 2x(x^2-y^2)}{(x^2+y^2)^2} = \frac{y(x^4-y^4)+4x^2y^3}{(x^2+y^2)^2}\\
	f^{'}_{y} &= x\frac{x^2-y^2}{x^2+y^2}+xy\frac{-2y(x^2+y^2) - 2y(x^2-y^2)}{(x^2+y^2)^2} = \frac{x(x^4-y^4)+4x^3y^2}{(x^2+y^2)^2}\\&\\
	\frac{\partial f}{\partial x}&(x, y) =
	\begin{cases*}
		\frac{yx^4-y^5+4x^2y^3}{(x^2+y^2)^2}, &\text{$x^2+y^2\neq 0$}\\
		0, &\text{$x^2+y^2=0$}
	\end{cases*}\\
	\frac{\partial f}{\partial y}&(x, y) =
	\begin{cases*}
	\frac{x^5-xy^4-4x^3y^2}{(x^2+y^2)^2}, &\text{$x^2+y^2\neq 0$}\\
	0, &\text{$x^2+y^2=0$}
	\end{cases*}\\&\\
	\frac{\partial^2f}{\partial y\partial x}&(0, 0) = \lim_{y\to 0}\frac{f^{'}_x(0, y) - f^{'}_x(0, 0)}{y} = \lim_{y\to 0}\frac{-y^5}{y^5} = -1\\
	\frac{\partial^2f}{\partial x\partial y}&(0, 0) = \lim_{x\to 0}\frac{f^{'}_y(x, 0) - f^{'}_y(0, 0)}{x} = \lim_{x\to 0}\frac{x^5}{x^5} = 1
	\end{align*}
	\[
	\frac{\partial^2f}{\partial{x}\partial{y}} \neq \frac{\partial^2f}{\partial{y}\partial{x}}
	\]
	\vspace{5mm}
	
	Из этих примеров видно, что в общем случае смешанные производные зависят от порядка дифференцирования.
	\vspace{5mm} 
	
	\textbf{Теорема:} Пусть в $\mathscr{U}(a) \subset \mathbb{E}^2$ определены $\frac{\partial^2f}{\partial x\partial y}$ и $\frac{\partial^2f}{\partial y\partial x}$, и эти производные непрерывны в точке $a = (a_1, a_2)$, тогда
	\[
	\frac{\partial^2f}{\partial{x}\partial{y}}(a) = \frac{\partial^2f}{\partial{y}\partial{x}}(a)
	\]
	
	\textbf{Доказательство:} Рассмотрим функцию
	\[
		U(x, y) = f(x, y) - f(x, a_2) - f(a_1, y) + f(a_1, a_2)
	\]
	Пусть $\Pi = \{(x, y) : |x - a_1|\leqslant r_1, |y-a_2| \leqslant r_2\}$, $\Pi \subset \mathscr{U}(a)$, где определены смешанные производные. Фиксируем $y\in (a_2-r_2, a_2+r_2)$ и на интервале $(a_1-r_1, a_1+r_1)$ Рассмотрим функцию 
	\[
		\varphi(x) = f(x, y) - f(x, a2)
	\]
	$\varphi$ дифф-ма на интервале $(a_1-r_1, a_1+r_1)$ и $U(x, y) = \varphi(x) - \varphi(a_1)$.
	Тогда, по теореме Лагранжа $\exists \Theta_1: 0<\Theta_1<1$:
	\[
		U(x, y) = \varphi^{'}(a_1 + \Theta_1\Delta x)\Delta x
	\]
	где $\Delta x = x-a_1$
	\[
		U(x, y) = [f^{'}_x(a_1+\Theta_1\Delta x, y)-f^{'}_x(a_1+\Theta_1\Delta x, a_2) ]\Delta x
	\]
	К выражению, стоящему в [...] применим теорему Лагранжа.
	
	 $\exists\Theta_2: 0<\Theta_2<1$:
	\[
		U(x, y) = f^{''}_{xy}(a_1+\Theta_1\Delta x, a_2+\Theta_2\Delta y)\Delta y\Delta x
	\]
	где $\Delta y = y-a_2$.
	\vspace{5mm}
	
	Аналогично фиксируем $x\in(a_1-r_1, a_1+r_1)$ и на интервале $(a_2-r_2, a_2+r_2)$ получаем
	\[
		U(x, y) = f^{''}_{yx}(a_1+\Theta_3\Delta x, a_2+\Theta_4\Delta y)\Delta y\Delta x
	\]
	
	\[
		f^{''}_{yx}(a_1+\Theta_3\Delta x, a_2+\Theta_4\Delta y) = f^{''}_{xy}(a_1+\Theta_1\Delta x, a_2+\Theta_2\Delta y)
	\]
	Учитывая непрерывность в точке $a$ при $\Delta x\to0, \Delta y\to0$, получаем $f^{''}_{xy}(a) = f^{''}_{yx}(a)$
	
	
	\textbf{Определение:} Функция $\omega = f(x, y)$ называется n раз дифференцируемой в точке $x = a\in \mathbb(E)^m$, если все ее частные производные порядка n-1 есть дифференцируемые функции
	\vspace{5mm}
	
	\textbf{Теорема:} (без доказательства) Пусть $\omega = f(x, y)$ дважды дифференцируема в точке $a$, тогда
	\[
		\frac{\partial^2f}{\partial{x}\partial{y}}(a) = \frac{\partial^2f}{\partial{y}\partial{x}}(a)
	\]
	
	\subsection*{Дифференциалы высших порядков. Отсутствие инвариантности их формы}
	
	\textbf{Определение:} Пусть $\omega = f(x)$ дважды дифференцируема в $D \subset \mathbb{E}^m$. $\forall x\in D $ $df(x) = \sum\limits_{j=1}^m\frac{\partial f}{\partial x_j}(x)dx_j$. Тогда дифференциалом 2 порядка будем называть
	\[
		d^2f(x) =d(df)(x) = \sum\limits_{j=1}^md\left(\frac{\partial f}{\partial x_j}\right)(x)dx_j= \sum\limits_{j=1}^m\left(\sum\limits_{k=1}^m
		\frac{\partial^2f}{\partial x_k\partial x_j}(x)dx_k\right)dx_j
	\]
	
	Дифференциалы высших порядков определяются таким же образом.
	\vspace{5mm}
	
	\textbf{Замечание:} Если рассмотреть дифференциал, как оператор
	\[
		d = \left(dx_1\frac{\partial}{\partial x_1}+\cdots+dx_m\frac{\partial}{\partial x_m}\right)
	\]
	
	То дифференциал n-ого порядка можно записать в виде
	\[
		d^2 = \left(dx_1\frac{\partial}{\partial x_1}+\cdots+dx_m\frac{\partial}{\partial x_m}\right)^n
	\]
	
	\textbf{Предложение:} Дифференциалы высших порядков не обладают свойством инвариантности формы.
	
	\textbf{Доказательство:} Пусть $\omega = f(x), $ $x_j=\varphi_j(t), $  $j=\overline{1, m}, $ $f, \varphi_j$ - дважды дифференцируемы.
	\[
		df(x) = \sum\limits_{j=1}^m\frac{\partial f}{\partial x_j}(x)dx_j, \text{ }dx_j = \sum\limits_{i=1}^m\frac{\partial \varphi_j}{\partial t_i}(i)dt_i
	\]
	
	\[
		d^2f(x) = \sum\limits_{k=1}^m\sum\limits_{j=1}^m
		\frac{\partial^2f}{\partial x_k\partial x_j}(x)dx_kdx_j+\sum\limits_{j=1}^m\frac{\partial f}{\partial x_j}(x)d^2x_j
	\]
	причем
	\[
		\sum\limits_{j=1}^m\frac{\partial f}{\partial x_j}(x)d^2x_j \neq 0
	\]
	
	\subsection*{Формула Тейлора для функций нескольких переменных}
	
	\textbf{Теорема: }[Разложение с остаточным членом в форме Лагранжа] Пусть функция $\omega = f(x)$ обладает непрерывными частными производными порядка n+1 в шаре $B_\delta(a)$, $\Delta x$ таково, что $a+\Delta x\in B_\delta(a)$. Тогда найдется $0<\theta<1$ такое, что 
	\[
		f(a+\Delta x) = f(a) + \sum\limits_{j=1}^n\frac{d^k f}{k!}(a)+r_{n+1}(\theta)
	\]
	где
	\[
		r_{n+1}(\theta) = \frac{d^{n+1}f(a+\theta \Delta x)}{(n+1)!}
	\]
	
	\textbf{Примечание:} $dx_j$ трактуется как $\Delta x_j$
	\vspace{5mm}
	
	\textbf{Доказательство:} $a+\Delta x\in B_\delta(a) \Rightarrow a-\Delta x\in B_\delta(a), $ $\forall t\in[-1, 1], a+t\Delta x\in B_\delta(a)$.
	
	\[
		f(a + t\Delta x) = f(a_1 + t\Delta x_1, \dots, a_m + t\Delta x_m) = \varphi(t)
	\]
	\[
		\varphi(0) = f(a)
	\]
	\[
		\varphi^{'}(t) = \sum\limits_{j=1}^m \frac{\partial f}{\partial x_j}(a+t\Delta x_j)\Delta x_j = df(a+t\Delta x)
	\]
	\[
		\varphi^{(k)}(t) = \sum\limits_{j_k=1}^m\cdots\sum\limits_{j_1=1}^m\frac{\partial^kf}{\partial x_{j_k}\dots\partial x_{j_1}}\Delta x_{j_1}\dots \Delta x_{j_k} = d^kf(a+t\Delta x)
	\]
	По формуле Тейлора
	\[
		\varphi(t) = \varphi(0) +\sum\limits_{k=1}^n\frac{\varphi^k(0)}{k!}t^k + r_{n+1}(\theta)
	\]
	где
	\[
		r_{n+1}(\theta) = \frac{\varphi^{(n+1)}(\theta t)}{(n+1)!}t^{(n+1)}
	\]
	Подставив $t=1$ получим требуемое равенство.
	\vspace{5mm}
	
	\textbf{Теорема: }[Разложение с остаточным членом в форме Пеано ](без доказательства) Пусть $f$ n-раз дифференцируема в точке $x =a$, тогда
	\[
		f(a+\Delta x) = f(a) + \sum\limits_{k=1}^n\frac{d^kf}{k!}(a)+o(\rho), \rho\to 0, \rho(\Delta x, 0)
	\]
	
		
\end{document}





















